\documentclass[14pt]{extarticle}
\usepackage[utf8]{inputenc}
\usepackage[polish]{babel}
\usepackage[T1]{fontenc}
\usepackage[a4paper, left=3.7cm, right=3.7cm, top=3cm, bottom=2cm]{geometry}
\usepackage{indentfirst}
\usepackage{graphicx}


\title{Opracowanie i implementacja modułu wykrywania obiektów ze strumieni video w systemie inteligentnego domu}
\author{Antoni Załupka}

\begin{document}
	\thispagestyle{empty}
	
	\begin{center}
		\Large{Wydział Nauk Ścisłych i Technicznych}
	\end{center}
	
	\vspace{2cm}
	
	\rule{\linewidth}{2mm} 
	
	\begin{center}
		\huge \textbf{Projekt WIG - sklep internetowy z ubraniami} \\
	\end{center}
	
	\rule{\linewidth}{0.5mm} 
	
	\vspace{2cm}
	
	\begin{center}
		\Large{Antoni Załupka}
	\end{center}
	
	\vspace{1.5cm}
	
	
	
	\vspace{8cm}
	
	\begin{center}
		\Large{2025, Uniwersytet Śląski}
	\end{center}
	
	\newpage
	
	\section{Wprowadzenie}
		Jako cel podsięwziąłem stworzenie prostego, lecz trzymającego się konkretnej estetyki sklepu internetowego. W sklepie dostępne będą ubrania z podziałem na kolekcje. Sklep będzie odnosił się do pracy licencjackiej, stworzonej na Uniwersytecie Jana Długosza w Częstochowie. Owocem tej pracy są ubrania z własnymi nadrukami, grafiki, szkice, profile na social mediach oraz magazyn modowy w formie pliku pdf. Marka nosi nazwę CHANDRA i cechuje się specyficznym stylem inspirowanym amerykańską kulturą Hip-Hop. Moim celem jest stworzenie strony internetowej z funkcją zakupów, przy zachowaniu charakterystycznego stylu CHANDRY. Projekt będzie wyróżniał sie sposobem przedstawienia ubrań. Zamiast tradycyjnych zdjęć, produkty będą prezentowane jako modele 3D, które użytkownik będzie mógł obracać i oglądać z różnych perspektyw. Reszta aplikacji będzie możliwie prosta, zachowując podstawowe funkcje sklepu internetowego.
		
		\begin{figure}[h]
			\centering
			\includegraphics[width=0.35\textwidth]{chandra-logo.png}
			\caption{Logo marki CHANDRA}
		\end{figure}

	\section{Opis ogólny}
		\subsection{Wymagania funkcjonalne}
		\begin{itemize}
			\item Przeglądanie produktów przez kolekcje.
			\item Karta produktu: zdjęcia, wariant (rozmiar/kolor), cena, dostępność, model 3D.
			\item Koszyk: dodaj/usuń, zmiana ilości, automatyczna suma.
			\item Zamówienie bez logowania (jako gość).
			\item Prosty admin panel z podglądem zamówień.
		\end{itemize}

		\subsection{Wymagania niefunkcjonalne}
		\begin{itemize}
			\item Responsywność (mobile-first).
			\item Podstawowa dostępność: alternatywne teksty, kontrast, obsługa klawiaturą.
			\item Wydajność: lazy-loading obrazów.
			\item Publikacja aplikacji w internecie z bezpiecznym protokołem HTTPS.
		\end{itemize}
		
	\section{Schemat podstron}
		\begin{itemize}
			\item Strona główna (/)
			\item Kolekcje (/collections)
			\begin{itemize}
				\item Szczegóły kolekcji (/collections/\{slug\})
				\begin{itemize}
					\item Produkt (/product/\{slug\})
				\end{itemize}
			\end{itemize}
			\item Zamówienie (bez logowania) (/checkout)
			\item Admin (/admin)
			\begin{itemize}
				\item Zamówienia (/admin/orders)
				\item Szczegóły zamówienia (/admin/orders/\{id\})
			\end{itemize}
			\item O marce (/about)
			\item Magazyn PDF (/magazine)
			\item Kontakt (/contact)
		\end{itemize}
	
	\section{Elementy interfejsu webowego}
		Do tworzenia elementów interfejsu wykorzystam bibliotekę Tailwind CSS, która umożliwia szybkie stylowanie przy pomocy klas.
		Dodatkowo, proste komponenty UI, takie jak przyciski, formularze czy nawigacja, zostaną zapożyczone jako gotowe komponenty z biblioteki Shadcn UI, opartej również na Tailwind CSS i Radix UI.
		

	\section{Wykorzystane technologie}
		\begin{itemize}
			\item \textbf{Next.js} — framework full‑stack (React, Node).
			\item \textbf{React} — biblioteka do budowy interfejsu użytkownika w podejściu komponentowym, podstawa Next.js.
			\item \textbf{Three.js} — renderowanie modeli 3D w przeglądarce za pomocą WebGL.
			\item \textbf{React Three Fiber} — integracja Three.js z React
			\item \textbf{Backend w Next.js} — API Routes/Route Handlers i Server Actions do obsługi koszyka, zamówień i panelu admina.
			\item \textbf{TypeScript} — statyczne typowanie i lepsza jakość kodu.
			\item \textbf{Tailwind CSS} — szybkie, responsywne stylowanie w podejściu mobile‑first.
			\item \textbf{SQLite} — lekka relacyjna baza danych (plikowa) na potrzeby projektu.
		\end{itemize}
		
	
\end{document}